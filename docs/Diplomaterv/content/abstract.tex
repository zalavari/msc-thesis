%\pagenumbering{roman}
\setcounter{page}{1}

\selecthungarian

%----------------------------------------------------------------------------
% Abstract in Hungarian
%----------------------------------------------------------------------------
\chapter*{Kivonat}\addcontentsline{toc}{chapter}{Kivonat}

A kvadratikus programozás a lineáris programozásnál egy általánosabb technika, hiszen megengedi négyzetes tagok jelenlétét a célfüggvényben. Ezzel az alkalmazások körét jóval kibővíti, ugyanakkor az általános feladat megoldása sokkal nehezebbé válik.

Ez a fajta optimalizációs technika többek között azért is érdekes és hasznos, mert ha a változók binárisak és nincsenek további korlátjaink, akkor a probléma megoldásához felhasználható egy kvantumállapotokat használó számítógép, ezzel remélhetőleg jelentősen csökkentve az optimalizáláshoz szükséges időt. A szakirodalom egyszerűen csak QUBO (Quadratic Unconstrained Binary Optimization) néven hivatkozik erre a fajta felírásra.

Gráfokon vágások keresése a (számítógép)hálózatok megjelenése óta egy sokat kutatott tématerület. Maximális vágást találni közismerten NP-nehéz probléma, ugyanakkor gyakorlati szempontból fontos, hiszen például a tipikus klaszterezési problémák megfogalmazhatók így, ha az adatot gráfként tudjuk reprezentálni.

A maximális vágással kapcsolatos problémákra többféle QUBO felírást is adok, melyeket elméleti és gyakorlati szempontból is összehasonlítóan elemzek. Az elkészített formulákat több szempontból elemzem, például a legegyszerűbb ilyen összehasonlítási metrika a felhasznált változók száma.


A QUBO-k optimalizálásához a  D-Wave Ocean nevű programcsomagját használtam fel, mely több lehetőséget kínál a formulák megoldására. A klasszikus megoldók mellett lehetőség van például valódi, a D-Wave Systems által forgalmazott kvantumszámítógépeket is használni, vagy a klasszikus és kvantum eljárásokat hibrid módon ötvözni.

Összehasonlítom a különböző lehetőségekből adódó módszereket azok eredményessége és hatékonysága alapján.
Természetesen a kapott eredményeket összevetem más, klasszikus heurisztikus megoldók használatával is.

A munka jelentős részét tette ki számos tapasztalat gyűjtése a D-Wave-es programcsomaggal kapcsolatban, hiszen a terület újdonságából kifolyólag az elérhető dokumentációk meglehetősen limitáltnak bizonyultak.

\vfill
\selectenglish


%----------------------------------------------------------------------------
% Abstract in English
%----------------------------------------------------------------------------
\chapter*{Abstract}\addcontentsline{toc}{chapter}{Abstract}

Kell angol fordítás?


\vfill
\selectthesislanguage

\newcounter{romanPage}
\setcounter{romanPage}{\value{page}}
\stepcounter{romanPage}