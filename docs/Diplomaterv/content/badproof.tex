
Ezt a következő tétel mondja ki, mely \az+\refstruc{3ORnotexist} általánosításaként fogható fel.

\begin{theorem}
	\label{MORnotexist}
	Az $x_1 \vee \dots \vee x_i \vee  \dots \vee x_n = t$ kifejezésnek legfeljebb $n-2$ darab ($y_1, \dots y_{n-2}$) segédváltozóval sem létezik kvadratikus függvénye, ha $n \geq 2$.
	
	Ismét tegyük fel indirekten, hogy létezik ilyen kifejezés, vagyis léteznek olyan együtthatók, melyre a (\ref{MORSUM}) kifejezés a logikai kapcsolat fennállása esetén $0$-t, ellenkező esetben valamilyen pozitív számot ad behelyettesítés után. A (\ref{MORSUM}) kifejezést úgy érdemes végiggondolni, hogy benne a következő tagok szerepelnek: $x_i$ változók egyedül, párban egymással, párban $t$-vel, $y_i$ változók egyedül, párban egymással, párban $t$-vel, $x_i$ és $y_j$ változók párban egymással, végül $t$ változó egyedül, és a konstans tag.
	
	\begin{align} \label{MORSUM}	
		\begin{split}		
		\sum_{i \in \left[ n \right] }{A_{i}x_i} + \sum_{\substack{ i,j \in [n] \\  i < j} }{A_{ij}x_i x_j} + \sum_{i \in \left[ n \right] }{A_{it}x_it} +
		\\
		\sum_{i \in \left[ n-2 \right] }{B_{i}y_i} + \sum_{\substack{ i,j \in [n-2] \\  i < j} }{B_{ij}y_i y_j} + \sum_{i \in \left[ n-2 \right] }{B_{it}y_it} + 
		\\
		  \sum_{\substack{ i \in [n] \\  k \in [n-2]} }{C_{ik}x_i y_k} + D_tt		 + C
		 \end{split}
	\end{align}

	A könnyebb jelölés kedvéért a (\ref{MORSUM}) kifejezés középső sorát rövidítsük $(B)$-vel.
	
	Hasonlóan \az+\refstruc{3ORnotexist} bizonyításához, mivel $0=0 \vee \dots \vee 0$ igaz, ezért $C=0$, így a továbbiakban ezt a konstans tagot elhagyhatjuk.
	
	Mivel ha egyetlen $x_i$ vagy $x_j$ változó igaz csak, akkor $t$ értéke is egyes kell legyen. Hasonló helyzet áll elő, ha $x_i$ és $x_j$ mindketten $1$-es értéket vesznek fel. Ezekből következik az alábbi három egyenlet tetszőleges $i < j$-re.
	
	\begin{align}
		A_i + A_{it} + \sum_{ k \in [n-2]}{C_{ik}y_k} + D_t + (B) =0 \label{MOR1} \\
		A_j + A_{jt} + \sum_{ k \in [n-2]}{C_{jk}y_k} + D_t + (B) =0 \label{MOR2} \\
		A_i + A_j + A_{ij} + A_{it} + A_{jt} + \sum_{k \in [n-2]}{C_{ik}y_k} + \sum_{k \in [n-2]}{C_{jk}y_k} + D_t + (B) = 0 \label{MOR3}
	\end{align}
	
	Ha (\ref{MOR1})-t és (\ref{MOR2})-t kivonjuk a (\ref{MOR3})-ből, akkor átrendezés után kapjuk a (\ref{MOR4}) azonosságot (tetszőleges $i \neq j$-re).
	
	\begin{align}
		A_{ij}-D_t-(B)=0 \label{MOR4}
	\end{align}

	Ha ezután $A_{ij}$ helyére $(D_t+(B))$-t helyettesítünk, akkor általánosan a (\ref{MORSUM2}) alakot kapjuk. 
	
	\begin{align} \label{MORSUM2}		
		\begin{split}		
			\sum_{i \in \left[ n \right] }{A_{i}x_i} + (D_t+(B))\sum_{\substack{ i,j \in [n] \\  i < j} }{x_i x_j} + \sum_{i \in \left[ n \right] }{A_{it}x_it} +
			\\
			+(B)+\sum_{\substack{ i \in [n] \\  k \in [n-2]} }{C_{ik}x_i y_k} + D_tt
		\end{split}
	\end{align}
	
	Most próbáljuk meg felírni a keletkező egyenletet ahhoz az esethez, amikor $x_i=1$ minden $i$-re, akkor egyszerűsítés után a (\ref{MORSUM3}) kifejezést kapjuk.
	
	\begin{align} \label{MORSUM3}		
	\begin{split}		
	0=\sum_{i \in \left[ n \right] }{A_{i}} + (D_t+(B))\left( \binom{n}{2}+1 \right)  + \sum_{i \in \left[ n \right] }{A_{it}}
	+ \sum_{\substack{ i \in [n] \\  k \in [n-2]} }{C_{ik}y_k}
	\end{split}
	\end{align}

	Ha \ref{MORSUM2} kifejezésből kivonjuk a \ref{MOR1} -et, minden $1 \leq i \leq n$  esetén, akkor a (\ref{MORSUM4}) kifejezést kapjuk.
	
	\begin{align} \label{MORSUM4}		
	\begin{split}		
	0=(D_t+(B))\left(\binom{n}{2}+1-n\right)
	\end{split}
	\end{align}

	Mivel $\left(\binom{n}{2}+1-n\right)$ valamilyen 0-nál nagyobb egész szám, leoszthatjuk vele az egyenletet, és ekkor azt kaptuk, hogy $(B)=-D_t$. Vagyis mivel $D_t$ egy konstans szám, $(B)$ helyettesítési értéke is mindig konstans. Ha ezt összevetjük a (\ref{MORSUM2}) kifejezéssel, akkor azt általános alakot immár még egyszerűbben írhatjuk, ahogyan a (\ref{MORSUM5}) alakban látható.
	
	\begin{align} \label{MORSUM5}		
		\begin{split}		
			\sum_{i \in \left[ n \right] }{A_{i}x_i} + \sum_{i \in \left[ n \right] }{A_{it}x_it} +	\sum_{\substack{ i \in [n] \\  k \in [n-2]} }{C_{ik}x_i y_k} + D_tt - D_t
		\end{split}
	\end{align}
	

	Mivel azt is tudjuk, hogy csak a $t$ változó egyedül nem lehet $1$ értékű, és így ebből az $D_t > 0$ egyenlőtlenség következik, ez pedig ellentmond az előbb kapott $A_t=0$ kifejezésnek. Ezzel az indirekt bizonyítás a végére ért, tehát az eredeti feltevésünk volt a helyes.

\end{theorem}