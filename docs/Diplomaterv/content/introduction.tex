%----------------------------------------------------------------------------
\chapter{\bevezetes}
%----------------------------------------------------------------------------

Gráfokon vágások keresése a (számítógép)hálózatok megjelenése óta egy sokat kutatott tématerület.

A polinom időben megoldható problémáktól, mint az egyszerű minimális vágás, a még közelítéssel
is nehezen kezelhető problémákig, mint a maximális vágás, széles a paletta. Ezáltal nem csak az elméleti eredmények érdekesek, hanem az alkalmazási területek is rendkívül sokrétűek. Például egy
klaszterezési probléma is megfogalmazható vágás kereséseként, ha az adatokat gráffal reprezentáljuk, úgy hogy az adatpontok a gráf csúcsai, az élek súlyát pedig valamilyen hasonlósági metrikából kapjuk. Ekkor a gráf k részre való vágása egy k klasztert előállító algoritmus.

A munkában először bemutatok általános optimalizációs technikákat, ezután több különböző vágási problémát vizsgálok meg, és elemzem, hogy milyen módon lehet az optimalizációs technikákat alkalmazni.
A munka végén pedig a gyakorlati eredményeket és tapasztalatokat gyűjtöm össze.